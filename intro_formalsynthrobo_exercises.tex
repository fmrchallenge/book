\documentclass{amsart}
\usepackage{hyperref}
\usepackage{fmrtexcommon}

\DeclareMathOperator{\Inf}{Inf}
\DeclareMathOperator{\Act}{Act}
\DeclareMathOperator{\ap}{AP}
\DeclareMathOperator{\true}{\mathtt{true}}
\DeclareMathOperator{\false}{\mathtt{false}}

\theoremstyle{definition}
\newtheorem{exer}{Exercise}


\begin{document}
\title{Introduction to Formal Methods in Robotics:\linebreak exercises}
\author{Scott C. Livingston}
\thanks{\textit{Email}: \href{mailto:slivingston@cds.caltech.edu}{slivingston@cds.caltech.edu}}
\date{\textbf{draft}; built \now}
\begin{abstract}
We present a collection of exercises that supplements the Introduction
\cite{ifmr2014}.  It should be possible to perform these without consulting
outside resources.
\end{abstract}
\maketitle


\begin{exer}
In Definition~1 we assumed that $\delta$ is not a partial function.  Why is this
without loss of generality?  ($f:A\rightarrow B$ is said to be a \textit{partial
  function} if there may be elements of $A$ for which $f$ is not defined.)
\end{exer}

\begin{exer}
Show that labeling in terms of a finite set of atomic propositions is equivalent
to labeling with variables that take values from finite domains.
(Cf.\ Section~5.)
%% By ``equivalence,'' we mean that for any transition system $T_1$ with labels
%% that are value assignments to a finite collection of variables, there is a
%% transition system $T_2$ with labels that are subsets of a set of atomic
%% propositions such that any property for $T_1$ can be
\end{exer}

\begin{exer}
Let $p$ and $q$ be atomic propositions.  Prove that $\Galways p \wedge \Galways q \equiv  \Galways \left( p \wedge q \right)$.  (Hint: Ignore details concerning the system exhibiting labels, and instead reason about infinite strings.)
\end{exer}

\begin{exer}
Construct two transition systems such that one simulates the other, but not the converse.
\end{exer}

\begin{exer}
Construct a parity game with two vertices in which exactly one vertex is winning
(for Player~0).
\end{exer}


\bibliographystyle{abbrv}
\bibliography{intro}
\end{document}
